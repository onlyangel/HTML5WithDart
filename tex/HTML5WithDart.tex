\documentclass[10pt]{book}
\usepackage[width=4.375in, height=7.0in, top=1.0in, bottom=0.7in, papersize={5.5in,7.0in}]{geometry}
\usepackage[pdftex]{graphicx}
\usepackage{amsmath}
\usepackage{amssymb}
\usepackage{tipa}
\usepackage{listings}
%\usepackage{txfonts}
\usepackage{textcomp}
%\usepackage{amsthm}
%\usepackage{array}
%\usepackage{xy}
\usepackage{fancyhdr}
\usepackage{lipsum}

\pagestyle{fancy}
\renewcommand{\chaptermark}[1]{\markboth{#1}{}}
\renewcommand{\sectionmark}[1]{\markright{\thesection\ #1}}
\fancyhf{}
\fancyhead[LE,RO]{\bfseries\thepage}
\fancyhead[LO]{\bfseries\rightmark}
\fancyhead[RE]{\bfseries\leftmark}
\renewcommand{\headrulewidth}{0.5pt}
\renewcommand{\footrulewidth}{0pt}
\addtolength{\headheight}{0.5pt}
\setlength{\footskip}{0in}
\renewcommand{\footruleskip}{0pt}
\fancypagestyle{plain}{%
\fancyhead{}
\renewcommand{\headrulewidth}{0pt}
}
%
%\parindent 0in
\parskip 0.05in
%
\begin{document}
\frontmatter
%
\chapter*{\Huge \center HTML5 with Dart }
\thispagestyle{empty}
%{\hspace{0.25in} \includegraphics{./ru_sun.jpg} }
\section*{\huge \center Create HTML5 Applications With Dart}
\section*{\small \center by Jose Angel Espinoza Portillo @OnlyAngel }
\newpage
\subsection*{\center \normalsize \copyright Jose Angel Espinoza Portillo 2013 }
\subsection*{\center \normalsize Some rights reserved. No part of this publication may be reproduced, stored in a retrieval system or transmitted in any form or by any means, electronic, mechanical or photocopying, recording, or otherwise for commercial purposes without the prior permission of the publisher.
HTML5 With Dart by Jose Angel Espinoza Portillo is licensed under a Creative Commons Attribution-NonCommercial-ShareAlike 4.0 International License. Based on a work at https://github.com/onlyangel/HTML5WithDart.}
\subsection*{\center \normalsize ISBN \dots}
\subsection*{\center \normalsize \dots Publications}
%
\newpage
\subsection*{\center \normalsize To Norita}
%
\chapter{Preface}
\lstinputlisting[language=Java]{code/test01.dart}
\section{Reason of the existence of this Book}

\section{Licences}
\section{Code}
\section{Subscription}
\section{How should read these book}
%
\tableofcontents
%
\mainmatter
%
\part{What is Dart?}
\chapter{Introduction}
\section{About Dart}
\subsection{Origin  (history)}
\subsection{Chrome vs v8}
\subsection{faster that v8}
\subsection{More than a language}
\section{Characteristics}
\subsection{j2js}
\section{Browser support}
\section{Tools}
\subsection{Editor}

\chapter{The actual language}
Here I will explain the language

\section{Dart Basics}
Here it come the basic part of the Dart language

\section{Comments}
Here I will explain the dart comment "paradigma"
\subsection{Line}
I Will explain the single line comments
\subsection{Multiline}
I will explain the multi line comments

%VARIABLES SECTION
\section{Variables \& Operations}
Variables are cool. Are even cooler when you usethem correctly. And don't get me started for operations.

\subsection{Basic Operations}
\subsubsection{Asigment}
yes im talking about equality
\subsubsection{Asert}
Yes these is an asert 
\subsection{Type of Variables}
\subsubsection{Numbers}
Number Variables are awesome, make me remember math and stuf.
\subsubsection{Booleans}
Boolean Variables are super cool. Just thing that is true is true and what is false is false.
\subsubsection{String}
Compared with both kind of variables previous to these. These is not that sexy. But is prety dammed usefull.
\subsubsection{String Operations}
And yes the Strings has its own operations section.
\subsubsection{Lists}
The lists are a kind of array but made Object and with all kind of extras. Pretty usefull too.
\subsubsection{Maps}
The headache of most languages efficiently speaking. These is not a problem for Dart =D.

% the ifs elses and all that stuff
\section{Flow Management Operations}
\subsubsection{conditionals}
the conditionals
\subsubsection{loops}
the loops
%\subsection{if else}
%\subsection{for}
%\subsection{do while}
%\subsection{break?}
%\subsection{switch case}



% Dont get me started with functiosn. Ok well lests start
\section{Functions}
Here I will explain what a function is.
\subsection{main()}
Main the Alpha of a Dart Application.
\subsection{The parameters}
Yes we use parameters and the parameters were before the HTML parameters.
\subsubsection{Default Values}
At the begining there was parameter, and after there exists the Default Values for Parameters
\subsubsection{Optional Parameters}
The roumors says that there are Optional Parameters behid those cool walls. Said the JS script.
\subsubsection{Named Optional Parameters}
And the callback JS method respond. Yes and some of them have names.
\subsubsection{Positional Params}
And the Dart came and show them the rigth order and position of the correct function calling.
\subsection{Function Operations}
The functions are so convenient that it even have it own Operations section.
\subsubsection{As an Object}
Yes you can have an object that represents a function and pass it as a parameter (YES A FUNCTION SENDED AS A PARAMETER OF OTHER FUNCTION, isn't that weird? But convenient).
\subsubsection{Closures}
Yeah! that Closures Thingi.
\subsubsection{ $=>$ }
Super fast way of, avoid type the keys.

%Classes
\section{Classes}
\subsection{Members}
(public variables)
\subsection{Class constructor}
And is not the guy that build the class. Or not exactly.
\subsubsection{Initializa list}
These wier animal
\subsubsection{Calling constructors}
Constructors callis other consturctors these is an bussiness now.
\subsection{Access Variables}
how to access variables 
\subsection{get and set}
Getters and Setters
\subsection{mixins}
\subsection{Operations with classes}

%libraries
\section{Libraries}
yes Libraries
\subsection{Make your own}
You can do it
\subsection{how to call them}
Hey you library come here
\subsubsection{Prefixes}
a prefix
\subsubsection{Partial Import}
Just you and you come here
\subsection{Included Libraries}
The libraries from dart
\subsubsection{Core}
The core
\subsubsection{Convert}
Convertion tools
\subsubsection{Html}
All the subject of the se book
\subsubsection{Async}
How to call remote stuff
\subsubsection{Futures}
Great functionality
\subsubsection{Isolates}
Mutli proceses
\subsection{pub.dartlang.org}
More stuff for free
\subsection{angular.dart}
Like Angular.js but with Dart
\subsection{and the list goes on and on}
Much more stuff for free



\backmatter
%
\begin{thebibliography}{99}
\bibitem{dartlang.org}
The official dart website
http://www.dartlang.org/

\end{thebibliography}
\end{document}
